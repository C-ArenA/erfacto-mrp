\section{Justificación}
La solución propuesta tiene 3 pilares importantes que la componen y muestran relevancia cada una por su cuenta:

\begin{itemize}
    \item La figura del paquete de software reutilizable.
    \item La figura del trámite en instancias gubernamentales y su modernización.
    \item El desarrollo de software libre.
\end{itemize}

De acuerdo a esto podemos ver la importancia de esta solución en distintos ámbitos:

\subsection{Tecnológico}
La creación de un paquete de software reutilizable para la creación y
seguimiento de trámites ayuda en el proceso de digitalización de uno de los
procedimientos más comunes en el ámbito público, brinda a los gobiernos la
posibilidad de aprovechar de mejor manera los datos resultantes de un trámite y
dan a la población herramientas que hacen más fáciles sus vidas.

Al usar máquinas de estados finitas logramos el aprovechamiento del conocimiento
en sistemas secuenciales digitales dentro del ámbito de sistemas de software
para lograr una abstracción del trámite.

El paquete no sólo facilitará la implementación de sistemas de software con
módulos de trámites, sino que de forma más directa simplificará la tarea de los
desarrolladores, siendo una pieza tecnológica dentro de proyectos más amplios.

\subsection{Económico}
El software reutilizable de código abierto suele se aprovechado hoy en día con el objetivo de
disminuir costos en la producción de sistemas gracias a su naturaleza de
comunidad y lo demandado de su funcionalidad.

De no existir el software reutilizable se deben invertir recursos para cada
proyecto que requiera la misma funcionalidad. Recursos que, de usar un paquete
de software, podrían conservarse, siendo que una parte del desarrollo ya estaría
implementada. Existen excepciones a este caso en proyectos con necesidades muy
específicas, pero en la mayoría de proyectos, una librería o framework bien
implementado son esenciales para disminuir costos de producción.

\subsection{Académico}
Los proyectos de software libre a nivel regional y a nivel universidad son
realmente escasos y al momento de redacción de este documento se desconoce de
algún caso de éxito en la carrera de Ingeniería Electrónica de la UMSA. 

Es por eso que este proyecto busca ser un punto de partida hacia la realización
de más proyectos del mismo tipo, es decir, de software libre reutilizable. Todo
esto a partir de esta travesía que a modo de ejemplo busca inspirar a más
estudiantes de la carrera.

\subsection{Político}
La realización de este proyecto no sólo se alinea con la necesidad de los
distintos gobiernos del mundo de digitalizar sus procesos administrativos - como
se puede constatar en Bolivia por el decreto xxx donde se solicita a las
distintas instituciones gubernamentales la realización de planes hacia un
gobierno electrónico -, sino además apunta a la preferencia que los gobiernos
tienen por el software libre, como se describe en el artículo xx de la ley xx.

\subsection{Social}
Los trámites son un dolor de cabeza para gran parte de la sociedad. Los
gobiernos hacen el intento por digitalizar los mismos y así aliviar a la
población, pero la existencia de herramientas reutilizables pueden acelerar
drásticamente este proceso. Se espera que más y más trámites sean digitalizados
en tanto más fácil sea hacerlo. De esta manera los distintos actores de la
sociedad podrán aprovechar las bondades de las tecnologías de la información.

La digitalización de trámites, que sería facilitada por este proyecto, mejora
además los mecanismos por los cuales la sociedad participa del gobierno. Impulsa
la apertura de la información de los gobiernos hacia la gente y se adecúa a las
nuevas necesidades de las personas.