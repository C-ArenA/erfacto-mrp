\section{Descripción de la Solución Propuesta}
Dada la problemática expuesta en la sección \ref{problem_statement}, en esta sección
se describe una primera aproximación a su solución de forma detallada.

Se propone desarrollar una herramienta a medida diseñada para satisfacer las necesidades específicas de la empresa de prefabricados ArenA en sus procesos productivos,
optimizando la gestión de inventarios, el cálculo de costos de fabricación y la planificación de la producción.

Este sistema implementará varias características comúnmente encontradas en un MRP (Material Requirements Planning), e incorporará también ciertas otras características existentes en un 
MRP II (Manufacturing Resource Planning) para cubrir procesos clave del negocio de manera sencilla y eficiente, en particular en el área de manufactura. 
Nótese, sin embargo, que ante el nivel de personalización y la no tan estricta definición de un MRP II, no se pretende ser una copia exacta de este tipo de sistemas, teniendo identificadas como necesarias de forma amplia las siguientes características.

\subsection{Características Principales}

\begin{enumerate}
  \item Gestión de Inventarios y Cálculo de Costos
        \begin{itemize}
          \item Registro y control de inventarios de materias primas, productos en proceso y productos terminados.
          \item Cálculo automatizado de costos de fabricación, basado en el consumo de materiales y las actividades registradas en el proceso de producción.
          \item Alertas automáticas ante niveles bajos de materias primas o cuando los almacenes de productos terminados estén al máximo de su capacidad.
          \item Registro de almacenes atómicos para un análisis detallado y fácil ubicación de artículos 
        \end{itemize}
        
  \item Abstracción de actividades principales del negocio
        \begin{itemize}
          \item Adaptación de actividades de compra, venta y fabricación en las distintas etapas productivas
          \item Creación de actividades personalizadas para nuevos procesos de fabricación
        \end{itemize}

  \item Registro de Compras y Ventas
        \begin{itemize}
          \item Registro detallado de compras de materias primas, asociado con información de proveedores.
          \item Control de envíos de productos terminados hacia las áreas de ventas, garantizando una integración fluida entre la producción y la comercialización.
        \end{itemize}

  \item Información sobre Proveedores y Compradores
        \begin{itemize}
          \item Base de datos para almacenar información clave sobre proveedores, como contactos, condiciones de entrega y tiempos de suministro.
          \item Base de datos de compradores con información de contacto y relación histórica con compras realizadas.
        \end{itemize}

  \item Generación de Reportes y Predicciones
        \begin{itemize}
          \item Reportes detallados sobre consumo de materias primas, costos de producción, inventarios y envíos.
          \item Reportes fáciles de emplear por el área contable en formato Excel.
          \item Predicciones automáticas sobre la demanda en base a datos históricos 
        \end{itemize}

  \item Interfaz Adaptada al Modelo de Negocio
        \begin{itemize}
          \item La interfaz estará diseñada para que los usuarios puedan registrar actividades específicas del modelo de negocio, sin necesidad de comprender conceptos técnicos contables como el de inventarios. Por ejemplo, al presionar un botón de "Quema en el horno de ladrillos", el sistema actualizará automáticamente los inventarios correspondientes (materias primas consumidas, productos en proceso y productos terminados).
        \end{itemize}
  \item Restful API para posibilitar la interoperabilidad
        \begin{itemize}
          \item Endpoints para registrar cambios en el inventario de manera automatizada desde sistemas de monitoreo o adquisición de datos como ser los sistemas SCADA (Que tengan habilitada la función de conexiones http)
          \item Endpoints para obtención de información relevante en el sitio principal de la empresa o en sus aplicativos de cálculo de materiales orientados al cliente
        \end{itemize}
  \item Registro de Maquinaria, Materiales y Mano de Obra
        \begin{itemize}
          \item Registro de máquinas con su consumo energético y otros datos relevantes como su estado de operación.
          \item Registro de materiales y su estado, para alertar sobre posibles necesidades de renovación.
          \item Registro de obreros con sus habilidades y competencias para optimizar la asignación a tareas específicas.
        \end{itemize}
\end{enumerate}

\subsection{Beneficios del Sistema}

Algunos de los beneficios que se conseguirán con este sistema son:

\begin{itemize}
\item Accesibilidad y Simplicidad: Una interfaz amigable y adaptada al flujo de trabajo real de la empresa, diseñada para usuarios sin formación técnica avanzada.

\item Automatización y Eficiencia: Reducción de errores y tareas manuales al integrar y automatizar procesos de inventario.

\item Asistencia en el cálculo de costos de producción basado en datos objetivos e históricos, respondiendo a cambios en el mercado de forma inmediata.

\item Toma de Decisiones Basada en Datos: Reportes y predicciones precisos que respaldan la planificación y optimización de recursos.

\item Personalización: Adecuación específica a las características y necesidades únicas del negocio, evitando la complejidad innecesaria de soluciones genéricas.
\end{itemize}

Con este sistema, la empresa podrá gestionar de manera eficiente sus inventarios y procesos de producción, alineándose con sus objetivos operativos y asegurando un uso óptimo de sus recursos.

% Arquitectura propuesta