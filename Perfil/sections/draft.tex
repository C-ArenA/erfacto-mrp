\section{Boceto del Perfil}

El cuadro \ref{table: boceto} sigue las recomendaciones de Escalera para la redacción de perfiles de proyecto \cite{escaleraTECNICASIDCIENCIAS2006}.

\begin{table}[htbp]
    \centering
    \begin{tabular}{p{0.25\textwidth}p{0.75\textwidth}}\toprule
        Pregunta & Respuesta \\ \midrule
        \textbf{Dónde}

        {\tiny Descripción de \textbf{escenario} universitario, comercial, industrial o social donde se realiza el proyecto} &

        El proyecto se aplicará en la empresa de prefabricados \mbox{\textbf{ArenA}}, principalmente para el área de producción. La misma cuenta con distintas plantas de fabricación ubicadas en las ciudades de La Paz y El Alto, Bolivia.\\

        \textbf{Qué asunto}

        {\tiny Descripción detallada del asunto o \textbf{problema principal} que requiere de estudio y del abordaje de solución del problema descrito} &

        Ante la falta de herramientas modernas de control de inventarios y de planificación de los procesos de fabricación se analizó la posibilidad de usar ERPs tradicionales, pero estos, al atender un amplio espectro de negocios, carecen de la suficiente personalización y facilidad de uso que requiere la empresa ArenA, sin contar que no todas sus funcionalidades son relevantes en esta primera etapa de digitalización de la fábrica, pero que por las cuales además los costos son muy altos y al pagarse por usuario registrado son bastante limitantes económicamente.

        Una posible solución es un sistema de software de tipo MRP, específico para la producción de prefabricados (en particular aquellos producidos por la empresa Arena) y que permita al menos 20 usuarios, sea accesible por internet, tenga una base de datos centralizada, permita la extensibilidad a nuevos procesos de producción y tenga endpoints que permitan la integración con sensores y otros sistemas actualmente usados por dicha empresa\\

        \textbf{Para qué}

        {\tiny Descripción de los propósitos generales, \textbf{metas} y objetivos específicos a ser logrados en el proyecto} &

        Controlar el uso de recursos en los procesos de producción, mejorar la toma de decisiones de la empresa, brindar herramientas que permitan reducir pérdidas y aumentar las utilidades. \\

        \textbf{Por qué}

        {\tiny Descripción de la pertinencia científica o tecnológica del asunto, y la \textbf{justificación} de metas que se quieren lograr una vez terminado el proyecto. Es también importante que se explique la importancia y rango de prioridad del proyecto dentro de los planes de desarrollo de una región del país} &

        Porque se desconoce información del uso de materias primas, cantidades producidas y flujos en la demanda, además, porque los costos son calculados de forma predictiva y no existen suficientes datos para poder calcularlos de manera real y efectiva. Se carece de herramientas para la toma de decisiones en la etapa de manufactura. \\

        \textbf{Cómo y con qué}

        {\tiny Descripción de la metodología (técnicas, procedimientos) y las herramientas que se utilizarán para realizar el estudio. Incluir también el plan de trabajo para la ejecución del estudio} &

        Se usarán una metodología híbrida ágil que tome aspectos de SCRUM y de RUP, con ciertos matices considerando un solo desarrollador. De esta manera, tomaremos aspectos de comunicación y creación de tareas como historias de usuario desde SCRUM y la organización de etapas de desarrollo para una buena planificación con generación de documentación UML desde RUP.

        El sistema, al requerirse accesible por internet utilizará tecnologías web, con arquitectura cliente-servidor, el cual buscará un enfoque modular, más cercano a un modelo monolítico que a uno de microservicios, entendiendo el tiempo destinado al desarrollo y la cantidad de desarrolladores y tomará en cuenta la integridad de datos, así como la escalabilidad del sistema. \\

        \textbf{Cuándo}
        {\tiny Descripción de las principales acciones requeridas para realizar el proyecto, incluyendo el tiempo que cada acción tomará para ser realizada} &

        Se plantea realizar el proyecto en el transcurso de 3 meses, dentro de los cuales nos apegaremos a las etapas indicadas por la metodología RUP y respondiendo a cada paso del proceso del software. \\
        \bottomrule
    \end{tabular}
    \caption{Boceto del Perfil de Proyecto}
    \label{table: boceto}
\end{table}
