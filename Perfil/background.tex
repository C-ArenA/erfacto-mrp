\section{Antecedentes}

\begin{tcolorbox}[breakable]
    
    Los antecedentes tienen que dar el origen que nos lleva a considerar el Problema
    que pretende ser resuelto por este proyecto. Introduce al mismo de forma
    algo más detallada que la introducción.
    Describe el escenario sobre el cual se desarrollará y mencionará qué se intentó para solucionarlo.
    
    Aquí puedo usar referencias al gobierno electrónico y las determinaciones
    del estado boliviano y de otras instituciones que suportean la idea de este
    proyecto.

    \begin{enumerate}
        \item Los trámites son importantes para el funcionamiento del estado
        \item Los gobiernos buscan digitalizarse
        \item Los gobiernos e instituciones prefieren software libre
        \item En Bolivia se hizo reglamentos
        \item Se adjudican obras para desarrollar software
        \item Se hizo el sistema SIAI con módulos de trámites
        \item Se hacen muchos desarrollos desde cero
        \item Existen librerías interesant
        \item Existe funcionalidad común 
        \item La reutilización de software es ventajosa
    \end{enumerate}
    
\end{tcolorbox}

Durante el desarrollo del sistema SIAI (Sistema de Información Ambiental
Industrial) del \say{Ministerio de Desarrollo Productivo y Economía Plural} por parte
de la empresa \say{2IES}, se pudo identificar ciertas funcionalidades comunes a muchos
sistemas de software gubernamentales, las cuales tienen que ver con los trámites, un proceso común en la administración pública de los gobiernos.

\subsection{El estado y el gobierno}

Si bien los temas filosóficos, históricos o políticos parecen carecer de
importancia en la presente propuesta, es importante entender el origen de la
problemática señalada más adelante desde la misma concepción del gobierno y del
estado, pues esto encaminará a entender su evolución, la cual eventualmente
origina la cuestión que se trata en este documento.

El estado tiene muchas definiciones desde distintos puntos de vista, como el
economicista de Marx o el histórico de Weber, quien denota cómo el estado es una
evolución que consta de 3 etapas. La primera de presión sobre la gente y las
siguientes como una suerte de sumisión mutuamente acordada por la población.

Sin embargo, para lo que nos atañe, el estado es un fenómeno político que supuso
la separación o la salida de lo político del terreno social y \textbf{la
conversión del individuo en un ciudadano}, cuya relación de pertenencia
fundamental será con el estado al margen de cualquier característica particular
\cite{gordilloperezestadosurge}. Esto implica que el ciudadano tiene en adelante
una serie de obligaciones para con el estado al cual pertenece.

% κυβερνέιν

Por su lado, uno de los pilares del estado (Además del territorio, la población, la soberanía y el derecho), el gobierno (del griego
$\kappa \upsilon \beta \varepsilon \rho \nu \acute{\epsilon} \iota \nu$
kybernéin \textquote{pilotar un barco} o \textquote{capitán de un barco}), es un
sistema orgánico de autoridades a través del cual se expresa el poder del
Estado, creando, afirmando y desenvolviendo el orden jurídico
\cite{jorgefernandezruizdederecho}. Es decir, es un actor que
funciona como cara del estado hacia la ciudadanía.

De esta manera, el gobierno es la entidad a la cual la ciudadanía se aproxima como figura de autoridad. Es el que administra mediante sus distintos brazos de acción.

\subsection{La Administración Pública}

La concepción de administración pública se remonta a muchos años atrás. Por
ejemplo, en el año 302 D.C. Claudio Mamertino ya hablaba del término
\say{administración de la cosa pública} (\textit{administratione res publicae})
\cite{nixonsaylorinpraiseroman}. Y es que esta estrechamente relacionada con el
ejercicio del gobierno. Sin embargo, la aplicación y definición de la misma no
estaría clara hasta la llegada de Bonnin, el padre de la ciencia de la
administración pública.

Bonnin, en 1808, daba una definición primigenia sobre la administración pública:
\textquote{La que tiene la gestión de los asuntos comunes respecto al ciudadano
como miembro del estado} \cite{omarguerrerobonninsigloxxi}.

De este modo, la administración pública conecta al gobierno con la ciudadanía en
busca de beneficiar a esta última y conectarla con el estado de forma inmediata.
Esta conexión se suele materializar en un procedimiento común, que es el
trámite.

\subsection{El trámite y la burocracia}

Poco se ha escrito estrictamente sobre los trámites a nivel académico. Sin
embargo, por conocimiento popular casi toda la sociedad sabe lo que son. Se sabe
también lo estrechamente relacionados que están con el gobierno, la burocracia y
el manejo de documentos.

La palabra trámite viene del latín \say{\textit{trames}},
\say{\textit{tramitis}}, que para los romanos significaba \say{senda},
\say{camino}, de donde se derivó el sentido actual de \say{vía legal o
procedimiento que debe seguir una gestión}. En inglés no existe una palabra que
tenga la traducción exacta de trámite, pero se usan distintas palabras como 
\say{\textit{Procedure}}, \say{\textit{Transaction}} o \say{\textit{Paperwork}}.
También en español se puede referir al mismo como \say{Procedimiento
administrativo}. En todas estas definiciones se tienen como común denominador
los términos: Proceso, Camino, Papeleo.

Los estados, para ejercer su poder mediante el gobierno, requieren de la
población cumplir con una serie de obligaciones, las cuales dieron paso al
surgimiento de diversas instituciones a las que acudimos a efecto de realizar
trámites.

Al respecto, de acuerdo a una definición del Gobierno de México en uno de sus
portales web se entiende como trámite a:

\begin{displayquote}
 Cualquier solicitud o entrega de información que las
personas físicas o morales del sector privado hacen ante una dependencia u
organismo descentralizado, ya sea para cumplir una obligación, obtener un
beneficio o servicio o, en general, a fin de que se emita una resolución, así
como cualquier documento que dichas personas estén obligadas a conservar
\cite{portalmexicotramite}.
\end{displayquote}

En el mismo portal se hace una clasificación de los trámites, donde se señalan 5
tipos de los mismos:

\begin{enumerate}
    \item De naturaleza obligatoria
    \item De beneficio
    \item De conservación
    \item De procedimiento
    \item De consulta
\end{enumerate}

Según la Real Academia de la Lengua Española, se define como \say{Cada uno de
los pasos y diligencias que hay que recorrer en un asunto hasta su conclusión}
\parencite{asaleDiccionarioLenguaEspanola}.

Los trámites constituyen además el conjunto de requisitos, pasos o acciones a
través de los cuales los individuos o las empresas piden o entregan información
a una entidad pública, con el fin de obtener un derecho o para cumplir con una
obligación \cite{rosethFinTramiteEterno2018}. A partir de esto, los distintos
trámites pueden tener algunas características distintas como poder ser gratuitos
o de pago o estar dirigido a sujetos específicos de la población (estar
condicionados).

Sin embargo, el proceso del trámite puede llegar a ser muy engorroso y
perjudicar a los ciudadanos. Tal es el caso de Domitila Murillo, quien debía
trasladarse entre varias ciudades de Bolivia para hacer largas filas y vagar
perdida entre una cantidad indefinida de requisitos. Cuando finalmente logró
recibir su cédula (el motivo del trámite), no le quedaron más que dos semanas
antes de fallecer \cite{charoskyquejatramite}

El caso de Domitila no es aislado, las distancias, la falta de definición de
requerimientos y la burocracia afectan al proceso del trámite. Los tiempos
suelen ser perjudiciales para la población, muchas veces llegando a tardar horas
por un simple trámite, como se puede ver en la figura \ref{fig:horastramite}.

\begin{figure}[htbp]
    \centering
    \includegraphics[width=0.7\textwidth]{assets/horastramite}
    \caption{Horas necesarias para completar un trámite, por país}{Fuente: Datos del Latinobarómetro, 2017}
    \label{fig:horastramite}
\end{figure}

El término burocracia se acuñó en el siglo 18 por el filósofo Francés Vincent de
Gournay, derivando del francés \textit{bureau} y \textit{cratie} que significan
\say{Escritorio para escribir} y \say{Gobierno} respectivamente
\cite{britannicabureaucracy}. Lo cual querría decir \say{Mandar desde el
escritorio}.

El sociólogo alemán Max Weber define burocracia como una forma racional de
organización que en su opinión es la forma más pira de sistema legal de
autoridad. La burocracia es según él, necesaria y algunas de sus características
fundamentales son las jerarquías, la especialización y la definición estricta
ded reglas y regulaciones \cite{brajnikdictionaryofpublicadmin}.

Sin embargo, esta forma de organización suele tener como consecuencia en la
práctica una connotación negativa, ya que implica el consumo de mayores tiempos
de ejecución para distintas tareas administrativas, como es el caso del trámite.
Una tendencia reciente cree que el uso de las tecnologías de la información
podrían ayudar a mitigar este problema.

\subsection{En busca del \textit{e-government}}

En una entrevista a Carlos Jiménez, responsable mundial de \textit{IEEE e-government}, el mismo señala que el gobierno electrónico es una fase para llegar a tener gobiernos inteligentes y abiertos y que consiste en implantar la tecnología para mejorar procesos administrativos y permitir la interacción con los ciudadanos.

Si bien lo anterior nos puede dar una idea sobre lo que es el Gobierno Electrónico, se debe notar que no existe consenso en su definición y que más de una vez el término se usa de forma indistinta con \say{Gobierno Digital} o incluso algunas veces con \say{Gobierno Inteligente}. Sin embargo, un factor común es el uso de tecnologías de la información dentro del gobierno, como veremos a continuación.

Las siguientes son algunas definiciones:

Según el Gobierno de México: El concepto de Gobierno Electrónico incluye todas aquellas actividades basadas en las modernas tecnologías informáticas, en particular Internet, que el Estado desarrolla para aumentar la eficiencia de la gestión pública, mejorar los servicios ofrecidos a los ciudadanos y proveer a las acciones de gobierno de un marco mucho más transparente que el actual.

Por su lado, el Gobierno de Bolivia define Gobierno Electrónico como: la aplicación de las tecnologías de la información y la comunicación (TIC) al funcionamiento del sector público, con el objeto de incrementar la eficiencia, la transparencia y la participación ciudadana.
Engloba la interacción digital entre el estado y los ciudadanos, entre entidades públicas, el Estado y los servidores públicos y, entre el Estado y las empresas, contribuyendo al uso intensivo de las TIC.

El Banco Mundial lo define como \say{El uso de las tecnologías de la información y comunicaciones para mejorar la eficiencia, la efectividad, la transparencia y la rendición de cuentas del gobierno}.

Las Naciones Unidas, por su lado, lo definen como \say{La utilización de Internet y la \textit{World Wide Web} para entregar información y servicios del gobierno a los ciudadanos}.

El elemento clave en estas definiciones es la \say{gestión pública por medios digitales}.

Muchos países han visto la transformación hacia el gobierno electrónico como una prioridad en los últimos años y para lograrlo han implementado políticas públicas y una serie de estrategias.

Si bien, el Gobierno Abierto no está necesariamente relacionado al Gobierno Electrónico, en la práctica se ha visto cómo ambos conceptos actúan de forma estrecha gracias al uso de las TIC. El Gobierno Abierto surge por la creencia que el acceso a la información de gobierno por parte de los ciudadanos es un derecho esencial que fortalece el ejercicio democrático \cite[13]{naserGobiernoElectronicoGestion2011}.

Dada la importancia que tiene la implementación del gobierno electrónico tanto para el estado como para la población, las Naciones Unidas realizan un reporte sobre los avances en la misma. Uno de los índices empleados es el \textit{EGDI (e-Government Development Index)}, que es un indicador importante y que nos permite ver la pronta adopción de políticas que favorecen la implementación del gobierno electrónico. Esto se puede ver reflejado en la figura \ref{fig:egdi2020_2022}, que compara la situación del indicador entre 2020 y 2022.

\begin{figure}[!h]
    \centering
    \includegraphics[width=0.7\textwidth]{assets/egdi2020_2022}
    \caption{Valores promedio del EGDI y sus componentes}{Fuente: 2020 and 2022 United Nations E-Government Surveys}
    \label{fig:egdi2020_2022}
\end{figure}

El Gobierno Electrónico brinda muchos beneficios a la población como la eliminación de barreras temporales y espaciales, acceso igualitario a la información, colaboración, aumento en la producción de bienes y servicios, en suma, brinda mayor calidad de vida a la ciudadanía.

Los beneficios se crean para los cuatro actores: Gobierno, Empresas, Ciudadanos y Empleados. Generando cuatro tipos de relaciones (figura \ref{fig:g2all}):

\begin{enumerate}
    \item G2C: Government to Citizen
    \item G2B: Government to Business
    \item G2E: Government to Employee
    \item G2G: Government to Government
\end{enumerate}

\begin{figure}[!h]
    \centering
    \includegraphics[width=0.7\textwidth]{assets/g2all}
    \caption{Modelo Relacional de Servicios de la Administración Pública}{Fuente: Naser, El Gobierno Electrónico}
    \label{fig:g2all}
\end{figure}

\subsection{FLOSS: Herramienta de libertad}

\subsection{Esfuerzos en Bolivia y la región}

\subsection{Sistema SIAI}

\subsection{RAI, IAA, DIA: Un caso común}

\subsection{Componentes de Software Reutilizables}


\subsection{Software Libre Reutilizable}

la creación del \textit{Sofware FOSS} se atribuye a Richard Stallman, conocido como el padre del código abierto. El mismo creía que todos merecían colaborar libre y abiertamente con otros utilizando software, por lo que en 1983 presentó el Proyecto GNU, el cual constituye el primer sistema operativo libre. Posteriormente en 1985 siguió con la creación de la Free Software Foundation para apoyar aún más a la comunidad del software libre. A finales de la década de 1990 el reconocimiento generalizado de Linux y el lanzamiento del código fuente del navegador \textit{Netscape} aumentó el interés y la participación en el Software libre. La etiqueta de \textit{Open Source} se creó en una sesión estratégica celebrada el 3 de febrero de 1998 en Palo Alto, California, poco después de que se publicara el código fuente de \textit{Netscape}. Sin embargo, el término que actualmente se considera más correcto de usar es el de \textit{FOSS}, ya que acuña ambas definiciones \textit{(Free and Open Source Software)}. 

El software libre es principalmente colaborativo y reutilizable, afectando positivamente a la economía. En un principio, las grandes corporaciones se negaban a apoyar el desarrollo de \textit{FOSS}. Sin embargo, vista la utilidad del mismo, hoy en día se prefiere su utilización.

El software que se reutiliza es software de mayor calidad y a menores costos económicos y temporales. Implica una gran ventaja sobre los desarrollos desde cero.