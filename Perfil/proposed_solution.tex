\section{Descripción de la Solución Propuesta}
Dada la problemática expuesta en la sección \ref{problem_statement}, en la que se
describe de manera breve el enfoque que se le piensa brindar, en esta sección se
trata de describir la primera aproximación a su solución de forma detallada.

Sin duda, la realización de una herramienta de software libre reutilizable para la
implementación de trámites y su correspondiente seguimiento a nivel de backend tiene
muchas ventajas, pero su realización no es del todo trivial y demanda que se
conceptualice en un modelo medianamente robusto, con metodologías claras y descripciones
escuetas.

Cuando uno piensa en trámites, usualmente piensa en burocracia, en largas filas,
una obligación muchas veces irrenunciable y una lista interminable de pasos a
seguir. Según la Real Academia de la Lengua Española, se define como "Cada uno de
los pasos y diligencias que hay que recorrer en un asunto hasta su conclusión"
\parencite{asaleDiccionarioLenguaEspanola}.

Si bien las definiciones de trámite son escasas y algunas pueden tratar su semántica
desde una perspectiva más funcional, es indudable que un trámite es un
procedimiento que consta de uno o más pasos a seguir. De este modo, se puede
vislumbrar una manera casi obvia de modelarlo usando conceptos de teoría
computacional, de matemáticas discretas o circuitos secuenciales. Esto es, usando
máquinas de estado finitas.

Este enfoque no es precisamente nuevo y ya se puede ver en una aproximación al modelado
en software de distintos trámites de la división de gestiones, admisiones y
registros de la UMSA, donde se emplearon máquinas de Turing que a efectos prácticos
se aproximan más a máquinas de estado finitas \parencite{nachoSISTEMACONTROLTRAMITES2007}.

Desde luego, las máquinas de estados finitas parecen ser una manera sencilla de modelar
los procesos de trámite a nivel de software, principalmente por su paralelismo
con los pasos y sus respectivas transiciones, asemejándose a un procedimiento administrativo,
como se puede evidenciar en la definición matemática de este autómata:

\begin{definition}[Máquina de Estados Finita]
	Una máquina de estados es un conjunto de 5 elementos $M=(S,I,O,v,w)$, donde
	$S$ representa a la colección de estados de $M$; $I$ representa al alfabeto de
	entradas para M; O es el alfabeto de salidas de M; $v:SxI->S$ es la función del
	siguiente estado; y $w:SxI->O$ es la función de salida \parencite{grimaldiDiscreteCombinatorialMathematics1998}.
\end{definition}

Este paralelismo es más notorio en un diagrama de estados como el que se muestra
en la figura \ref{}, donde a cada estado se le puso un nombre cercano a la
naturaleza de los procedimientos administrativos de forma adrede para hacer más
obvia la relación.