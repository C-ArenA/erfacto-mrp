\section{Temario}

Para el temario del documento final de proyecto se considerará una estructura que describa bien la naturaleza del proyecto y que además se adecúe a este que, después de todo, será un producto de software.

De acuerdo a la ingeniería de software, el software tiene un ciclo de vida o un \say{proceso del software}, el cual se modela de acuerdo a la metodología de desarrollo sobre la cual se realice. Sin embargo, varios autores concuerdan en que existen ciertas etapas estructurales ajenas a cualquier metodología. Según Pressman \cite[13]{pressmanSoftwareEngineeringPractitioner2010}, estas etapas serían:

\begin{itemize}
    \item Comunicación.
    \item Planeación.
    \item Modelado.
    \item Construcción.
    \item despliegue.
\end{itemize}

Por su lado, Sommerville las simplifica en 4 etapas:

\begin{itemize}
    \item Software Specification.
    \item Software Development.
    \item Software Validation.
    \item Software Evolution.
\end{itemize}

El temario del documento final del proyecto obedecerá entonces a esta definiciones del proceso de software para no depender de la metodología usada.

Nótese, sin embargo, que este temario es tentativo, lo cual quiere decir que pueden surgir cambios durante la realización del proyecto.

\begin{itemize}
    \item Presentación
    \item Agradecimientos
    \item Resumen
    \item Índice
    \item Glosario
\end{itemize}

\begin{syllabus}
    \item \textbf{Generalidades del proyecto:} Describirá los antecedentes del proyecto, así como la problemática que se ha identificado, para la cual se plantea una solución a través del objetivo. De igual modo se hará referencia a la justificación del proyecto y los alcances y límites que se plantearon durante su gestación.
    \item Revisión de la literatura
    \item Definición de requerimientos
    \item Modelado y conceptualización
    \item Implementación
    \item Pruebas de Validación
    \item Despliegue
    \item Integración en el SIAI
    \item Resultados y conclusiones
\end{syllabus}

\begin{itemize}
    \item Bibliografía y Referencias
    \item Anexos
\end{itemize}